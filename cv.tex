%%%%%%%%%%%%%%%%%%%%%%%%%%%%%%%%%%%%%%%%%
% Medium Length Professional CV
% LaTeX Template
% Version 2.0 (8/5/13)
%
% This template has been downloaded from:
% http://www.LaTeXTemplates.com
%
% Original authors:
% Rishi Shah 
% Jan Kuzel
%
% Important note:
% This template requires the cv.cls file to be in the same directory as the
% .tex file. The cv.cls file provides the resume style used for structuring the
% document.
%
%%%%%%%%%%%%%%%%%%%%%%%%%%%%%%%%%%%%%%%%%

%----------------------------------------------------------------------------------------
%	PACKAGES AND OTHER DOCUMENT CONFIGURATIONS
%----------------------------------------------------------------------------------------

\documentclass{cv} % Use the custom resume.cls style
\usepackage{hyperref}
\usepackage{xcolor}
\hypersetup{
    colorlinks,
    urlcolor={blue!50!black}
}
\usepackage[T1]{fontenc}
\usepackage[utf8]{inputenc}
\usepackage[left=0.5in,top=0.5in,right=0.5in,bottom=0.5in]{geometry} % Document margins
\newcommand{\tab}[1]{\hspace{.2667\textwidth}\rlap{#1}}
\newcommand{\itab}[1]{\hspace{0em}\rlap{#1}}
\name{Jan Ku\v zel} % Your name
\address{\href{http://linkedin.com/in/jan-ku\%C5\%BEel-685330116}{linkedin.com/in/jan-kuzel-685330116}}
\address{\href{http://www.github.com/honzakuzel1989}{github.com/honzakuzel1989}}
\address{\href{mailto: honza.kuzel.1989@gmail.com}{honza.kuzel.1989@gmail.com}}

\begin{document}

%----------------------------------------------------------------------------------------
%	EDUCATION SECTION
%----------------------------------------------------------------------------------------

\begin{rSection}{Education}

{\bf Faculty of Information Technology, Brno University of Technology} \hfill {\em 2011 - 2013} 
\\ M.Sc., Information Technology Security

\\\\{\bf Faculty of Information Technology, Brno University of Technology} \hfill {\em 2008 - 2011} 
\\ B.Sc., Information Technology

\end{rSection}

\begin{rSection}{Career Objective}
 To work for an organization which provides me the opportunity to improve my skills which i can use for achieve organizational objectives.
\end{rSection}
%--------------------------------------------------------------------------------
%    Projects And Seminars
%-----------------------------------------------------------------------------------------------
\begin{rSection}{Projects}
\begin{rSubsection}
{\href{https://www.vfnuclear.com/en/products/waste-assay-monitor}{Waste Assay Monitor, Tianwan Nuclear Facility, China}}{}
{Team leader for the control application and integration between software and hardware}{}

Quantitative and qualitative characterization of radioactive waste with total activity in a barrel. The sophisticated evaluation allows the total barrel activity to be evaluated, including its distribution in the barrel.
\end{rSubsection}

\begin{rSubsection}
{\href{https://www.vfnuclear.com/en/free-release-monitor-frm-03-delivery-for-chernobyl-npp}{Free Release Monitor, Chernobyl Nuclear Power Plant, Ukraine}}{}
{Main developer for the key part of project, calculations integration and design of database}{}

A comprehensive monitoring equipment for measuring the activity of radionuclides in waste materials before they are released into the environment.
\end{rSubsection}

%\begin{rSubsection}
%{\href{https://www.vf.cz/files/upload/reference/SEJVAL.png}{Central Information System of Radiation %Technology, Dukovany Power Plant, Czech}}{}
%{Main developer for central application used for monitoring and control devices in facility}{}

%The monitoring and control system for the nuclear facility collects and visualize data from the hundreds %of places and allowed to control devices in that facility.
%\end{rSubsection}

\begin{rSubsection}
{\href{https://www.vf.cz/en/products/dars-control-system-for-the-calibration-laboratory-dars}{Control System for the Calibration Laboratory, Worldwide}}{}
{Developer for the part of application for the general communication with hardware}{}

The control system for the calibration laboratory is a complete system designed for the overall operational provision of calibration laboratories that provide ionising radiation instrument calibration.
\end{rSubsection}

\begin{rSubsection}
{\href{https://play.google.com/store/apps/details?id=eu.jksoft.planningcalendar}{Working hours calendar}}{}
{Creator and maintainer of the application with whole lifecycle for OS Android}{}

The application is the smart electronic alternative to the classic paper planning calendar which is well known from diaries.
\end{rSubsection}

\begin{rSubsection}
{\href{http://honzakuzel.eu/mwarrayextensions.html}{MWArray Extensions}}{}
{Developer of whole tested, benchmarked and published library}{}

The library was used like the automatic mapper between different type of classes in MATLAB runtime and C\#.
\end{rSubsection}

\end{rSection}

%----------------------------------------------------------------------------------------
%	WORK EXPERIENCE SECTION
%----------------------------------------------------------------------------------------

\begin{rSection}{Work Experience}

\begin{rSubsection}{\href{https://www.vfnuclear.com/en/}{VF, a.s., \v Cern\' a Hora}}{\em 2013 - now}{Software developer}{TFS, Git, VS, .NET, C\#, WPF, TPL, LINQ, Oracle, Bash, Python, C}
Real projects usually on the field of radiation technology long-term in service on the cities around the world.
\end{rSubsection}

\begin{rSubsection}{\href{http://www.fask.cz/}{Fask, s.r.o., Brno}}{\em 2011 - 2013}{Programmer}{.NET, C\#, WinForms, Web services, VS, MS SQL Server, SVN}
The project for the warehouse management for a few companies in Czech Republic.
\end{rSubsection}

\end{rSection}
\end{document}
